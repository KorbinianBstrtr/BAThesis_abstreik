% !TEX root = ../main.tex
\label{section:introduction}
\section{Introduction}

In recent years, deep learning models have proven to perform well in a wide range of applications, many of which include working with real world data. For example, Neural Networks have successfully been applied to the task of image classification \cite{NIPS2012_4824}, modeling fluid dynamics \cite{DBLP:journals/corr/SinghMD16, Tompson:2017:AEF:3305890.3306035} and simulating heat exchanger performance \cite{doi:10.1080/10789669.1999.10391233}. Traditionally, such tasks have either been impossible or otherwise simulated by hand-crafted models \cite{Cursi2005PhysicallyCN}. The latter were designed in a way to align with physical principles, such as gravity, the conservation of energy, or Newton's laws. This leads to high precision and avoids the problem of generalisation, since the theories are assumed to be true accross the whole domain space. Nevertheless, manually building models requires intense effort and domain knowledge, and their performance may suffer from flawed parameter estimates or theories. In contrast, deep learning models do not rely on correctly understanding the complex underlying system structure, but instead simply learn the behaviour from previously observed data. Furthermore, data-driven models outperform hand-crafted models in speed, since the output is computed by calculating a series of matrix multiplications and applying activation functions at every layer. Moreover, data-driven models can be applied even when no scientific theories are known. However, utilising deep learning models comes at the high cost of requiring large amounts of training data for data-driven models to create realistic predictions aligning with physical principles. In addition, the data has to be spread accross the whole domain space, which is often not the case. A lack of data in one region will already lead to unrealistic model predictions in that area, since the deep learning model does not specifically learn the physical rules and can therefore not generalise well.\\
\indent This raises the question, whether we can combine the scientific knowledge and the power of deep learning models in order to achieve more realistic results. Instead of purely relying on data, we can actively train the network to align its predictions with known physical principles. This could improve generalisation, help to create more realistic predictions and decrease the required amount of training data.\\

\clearpage

